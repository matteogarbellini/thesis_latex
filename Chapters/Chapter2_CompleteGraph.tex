\section{Results for the complete graph}
We now turn our attention to the complete graph. As we discussed in the preliminaries in \Cref{subsec:search qw complete graph} and \Cref{subsec:local adiabatic}, with the complete graph we are able to solve the search problem using the standard time-independendent quantum walks Hamiltonian with time scaling of $O(\sqrt{N})$. Additionally for the unstructured search - which is equivalent to a search on the complete graph - we showed that with the local adiabatic evolution it is possible to get the same speedup of $O(\sqrt{N})$, while that was not the case for the global adiabatic evolution that had the same time scaling as the classical search. Although Wong proved that for the complete graph it is not possible to solve the search problem with an adiabatic quantum walk algorithm, for completeness we extend the time-dependent Hamiltonian implementation to the complete graph. Clearly we are not able to achieve any speedup nor necessarily any comparable time scaling, but we might get some interesting insights in terms of probability distribution and robustness.

    \subsection{Comments on the placement of $\gamma$}
    Firstly we recall that in \Cref{subsec:complete graph} we introduced the search Hamiltonian as
    \begin{equation}
      H = \gamma L -\ket{w}\bra{w}
    \end{equation}
    We quickly notice that the $\gamma$ parameter is in front of the Laplacian, compared to the time-dependent Hamiltonian considered throughout our work where $\gamma$ was placed in front of the oracle $\ket{w}\bra{w}$. We could now proceed by comparing the time-independent and time-dependent approaches with the $\gamma$ placement as in literature - i.e. in front of the Laplacian - or as we did in \Cref{sec:time_dependent_Hamiltonian}. In order to be consistent with the standard quantum walks search on the complete graph discussed in \Cref{subsec:search qw complete graph} we proceed by considering the following Hamiltonians, with $\gamma$ in front of the Laplacian. This ensures that the time-dependent probability distribution is compatible with the time-independent one, and does not require to re-evaluate the optimal $\gamma$ for the time-independent approach. Nevertheless the two approaches can be considered equivalent. The Hamiltonians are given by
    \begin{equation}
      H = \gamma L-\ket{w}\bra{w}\hspace{45pt}\mbox{(independent)}
    \end{equation}
    \vspace{-0.5cm}
    \begin{equation}
      H(s) = (1-s)\gamma L - s\ket{w}\bra{w}\hspace{20pt}\mbox{(dependent)}
    \end{equation}\\


    \noindent
    Having defined the Hamiltonians, we proceed as in the previous section, following however a more qualitative approach. In order to do so we compare the probability distribution using the heatmap plots introduced in \Cref{subsec:time_independent_benchmarks, subsec:time_dependent_results}.


    \subsection{Probability distribution and qualitative robustness}
    Recalling that for the time-independent quantum walks implementation the optimal $\gamma$ is given by $\gamma=1/N$, we evaluate the probability for $\gamma$ in a neighbourhood of $1/N$ and up to $T=N$. This allows us to have a complete picture of the probability distribution, as can be seen in the following heatmap plots.

    %%
%COMPLETE GRAPH LOCALIZATION HEATMAP
%%

    As previously discussed we are able to solve the search problem with the time-independent algorithm in a time of the order of $T=\pi/2\sqrt{N}$. Indeed we see from the figure above that the probability is close to $p=1$ for $T=11$, as expected. On the other hand the time-dependent approach is able to solve the search with a single iteration for $T\rightarrow N$, showing no significant speedup compared to the classical search. This however leaves space for a multiple run search as introduced in \Cref{subsec:multiple_runs}. We therefore investigate this possibility by evaluating the maximum probability found for $T=\pi/2\sqrt{N}$ with the time-dependent algorithm and study the \textbf{iterations} distribution.

    %%
%T - ROBUSTNESS
%%
\begin{figure}[ht]
  \centering
  \begin{tikzpicture}

    \begin{axis}[name=plot,
    xmin=0, xmax=110,
    ymin=0, ymax=7,
    width = 100mm,
    xlabel = Dimension $N$,
    ylabel = Iterations $I$]
    \addplot[color=rossoscuro,mark=*,mark size=2.5pt, x=a, y=b] table{./Data/fig18/iterations.txt};\addlegendentry{iterations}
    \end{axis}\label{plot_one}

    \begin{axis}[name=plot,
    axis y line*=right,
    axis x line=none,
    xmin=0, xmax=110,
    ymin=0, ymax=1,
    width = 100mm,
    xlabel = Dimension $N$,
    ylabel = Probability $p$]
    \addlegendimage{mark=*,mark size=2.5pt, color=rossoscuro}\addlegendentry{Iterations}
    \addplot[color=verdescuro,mark=*,mark size=2.5pt, x=a, y=b] table{./Data/fig18/probability.txt};\addlegendentry{Probability}

    \end{axis}


  \end{tikzpicture}
  \caption[]{\textbf{Iterations \bm{$I$} and probability \bm{$p$} for the multiple run search on complete graph.} The plot shows the probability $p$ and the number of iterations $I$ for the multiple run search with the time-dependent Hamiltonian. The time is constrained to $\pi/2\sqrt{N}$ as in the solution of the standard quantum walks search. It is clear that the number of iterations increases linearly, although very slowly. Notice in fact that $N$ goes all the way up to $N=101$. For the limit of large $N$ the time-dependent approach is therefore not a valid alternative, nor of comparable performance.}
  \label{fig:iterations_complete_graph}
\end{figure}


    It is clear from the plot that the number of run iterations increases with the graph size linearly, although very slowly (notice that $N$ goes up to 101). In the limit for large $N$ the multiple runs search does not improve the time-dependent approach, therefore the standard quantum walks search remains the strongest option.
    \\


    \noindent
    However, we can now address the \textbf{robustness} properties of the two approaches. Similarly to the cycle graph, the complete graph has a probability distribution with regions of high and low probability when considering the time-independent Hamiltonian. Compared to the smooth probability distribution of the time-dependent algorithm we can safely say that the latter is more robust than the first, both for noise on $T$ and $\gamma$. It is to be noted however that for the cycle graph the time-independent and time-dependent approaches are of comparable performance, therefore the robustness has a great impact to the overall performance. In this scenario the time-independent approach performs better in terms of probability, and having less robustness does not justify the \textit{almost} linear time-scaling of the time-dependent one. \\


    \noindent
    Although the time-dependent algorithm is not able to perform as well as the time-independent one, striking is the difference in probability distribution between the non-linear $s_{NL}(t)$ and the linear $s_L(t)$, showing the importance of the shape of the interpolating schedule. From the following plots we can see that the time-dependent Hamiltonian with the linear interpolating schedule $s_L(t)$ is not able to solve the search for $T=N$. The improvement on the performance is therefore achieved by findind the optimal interpolating schedule.

    %%
%COMPLETE GRAPH HEATMAP PLOTS - ROLAND-CERF and LINEAR
%%

\begin{figure}[ht]
  \centering
  \subfloat[][$s_{NL}(t)$]
  {
    \hspace{-0.8cm}
    \begin{tikzpicture}
      \begin{axis}[name=plot,
      zmin=0,zmax=1,
      view={0}{90},
      colormap={inferno}{rgb255=(248,251,155) rgb255=(251,179,21) rgb255=(236,103,38) rgb255=(187,54,84) rgb255=(119,29,109) rgb255=(49,9,92) rgb255=(1,1,8)},
      colorbar,
      colorbar right,
      point meta min=0,
      point meta max=1,
      width = 65mm,
      height= 65mm,
      xlabel = $\bm{T}$,
      ylabel = $\bm{\gamma}$
      ]
      \addplot3 [surf] table[x=a,y=b,z=c]{./Data/fig17/51_cerf_complete.txt};

      \end{axis}
    \end{tikzpicture}
  }
  \subfloat[][$s_L(t)$]
  {
    \begin{tikzpicture}
      \begin{axis}[name=plot,
      zmin=0,zmax=1,
      view={0}{90},
      colormap={inferno}{rgb255=(248,251,155) rgb255=(251,179,21) rgb255=(236,103,38) rgb255=(187,54,84) rgb255=(119,29,109) rgb255=(49,9,92) rgb255=(1,1,8)},
      point meta min=0,
      point meta max=1,
      width = 65mm,
      height= 65mm,
      xlabel = $\bm{T}$,
      ylabel = $\bm{\gamma}$,
      ylabel near ticks, yticklabel pos=right
      ]
      \addplot3 [surf] table[x=a,y=b,z=c]{./Data/fig17/51_linear_complete.txt};
      \pgfplotsset{contour/every contour label/.style = {
                 sloped,
                 inner sep=2pt,
                 transform shape,
                 every node/.style={mapped color,fill=none},
                },}
      \addplot3[
              contour gnuplot=
              {
              draw color=white,
              levels={0.5},
              labels=true,
              contour label style={nodes={text=white,font={\large}, yshift=-2.5ex, xshift=3.0ex}},
              handler/.style=smooth},
              contour/label distance=300pt,
              line width=1.5pt,
              contour/labels over line
             ] table [
                      x = a,
                      y = b,
                      z = c,
                     ]{./Data/fig17/51_linear_complete.txt};
      \end{axis}
    \end{tikzpicture}
  }

  \caption[]{\textbf{Probability distributions for the complete graph \bm{$C(51)$} with the time-dependent Hamiltonian.} The figure shows the probability distribution for a complete graph of $N=51$ using the time-dependent Hamiltonian with the non-linear $s_{NL}(t)$ (a) and linear $s_L(t)$ (right) interpolating schedules. It illustrates the great impact of the interpolating schedule on the overall performance of the time-dependent algorithm, where the choice of $s(t)$ is critical.}
  \label{fig:heatmap-complete}
\end{figure}

