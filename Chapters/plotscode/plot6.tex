%%
%PROBABILITY DISTRIBUTION FOR SAMPLED N, UP TO P=1
%%

\begin{figure}[ht]
\centering
  \begin{tikzpicture}
    \begin{axis}[name=plot,
    xmin=0, xmax=441,
    ymin=0, ymax=1,
    width = 120mm,
    height=95mm,
    xlabel = $\bm{T}$,
    ylabel = $\bm{p}$,
    legend style={at={(0.95,0.05)},
    anchor=south east}]
    \addplot[color=verdescuro,no marks, thick, x=a, y=b] table{./Data/fig6/fig6_21_lin.txt};\addlegendentry{$N=21$, $\gamma=1.5$, $s_L(t)$}
    \addplot[color=verdescuro,dashed, no marks, thick, x=a, y=b] table{./Data/fig6/fig6_21_cerf.txt};\addlegendentry{$N=21$, $\gamma=1.05$, $s_{RC}(t)$}

    \end{axis}
  \end{tikzpicture}
  \caption{\textbf{Probability distibution for a $\bm{Cy(21)}$ with sampled $\bm{\gamma}$}: The figure shows the probability distribution for the cycle graph $Cy(21)$, evaluated with the time-dependent hamiltonian using the interpolating schedules (solid) linear $s_L$ and (dashed) Roland-Cerf $S_{RC}$. We can see that the probability increases with time, as expected. }
  \label{fig:probability_sampled_gamma}
\end{figure}
