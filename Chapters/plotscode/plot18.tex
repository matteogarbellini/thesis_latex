%%
%T - ROBUSTNESS
%%
\begin{figure}[ht]
  \centering
  \begin{tikzpicture}

    \begin{axis}[name=plot,
    xmin=0, xmax=110,
    ymin=0, ymax=7,
    width = 100mm,
    xlabel = Dimension $N$,
    ylabel = Iterations $I$]
    \addplot[color=rossoscuro,mark=*,mark size=2.5pt, x=a, y=b] table{./Data/fig18/iterations.txt};\addlegendentry{iterations}
    \end{axis}\label{plot_one}

    \begin{axis}[name=plot,
    axis y line*=right,
    axis x line=none,
    xmin=0, xmax=110,
    ymin=0, ymax=1,
    width = 100mm,
    xlabel = Dimension $N$,
    ylabel = Probability $p$]
    \addlegendimage{mark=*,mark size=2.5pt, color=rossoscuro}\addlegendentry{Iterations}
    \addplot[color=verdescuro,mark=*,mark size=2.5pt, x=a, y=b] table{./Data/fig18/probability.txt};\addlegendentry{Probability}

    \end{axis}


  \end{tikzpicture}
  \caption[]{\textbf{Iterations \bm{$I$} and probability \bm{$p$} for the multiple run search on complete graph.} The plot shows the probability $p$ and the number of iterations $I$ for the multiple run search with the time-dependent Hamiltonian. The time is constrained to $\pi/2\sqrt{N}$ as in the solution of the standard quantum walks search. It is clear that the number of iterations increases linearly, although very slowly. Notice in fact that $N$ goes all the way up to $N=101$. For the limit of large $N$ the time-dependent approach is therefore not a valid alternative, nor of comparable performance.}
  \label{fig:iterations_complete_graph}
\end{figure}
