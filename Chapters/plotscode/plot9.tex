%%
%TAU DISTRIBUTION FOR INCREASING T_MIN
%%
\begin{figure}[ht]
  \centering
  \begin{tikzpicture}
    \begin{axis}[name=plot,
    xmin=0, xmax=60,
    ymin=0, ymax=630,
    width = 100mm,
    xlabel = Lower bound time \bm{$T_{\min}$},
    ylabel = $\bm{\tau}$ ]
    \addplot[color=rossoscuro,mark=*,mark size=2.5pt, x=a ,y=b] table{./Data/fig9/51_tau_lbt_static.txt};\addlegendentry{time-independent}
    \addplot[color=verdescuro,mark=*,mark size=2.5pt, x=a,y=b] table{./Data/fig9/51_tau_lbt_linear.txt};\addlegendentry{$s_L(t)$}

    \end{axis}
  \end{tikzpicture}
  \caption{\textbf{\bm{$\tau$} for increasing lower bound time \bm{$T_{\min}$}, $\bm{Cy(51)}$.} The plot shows $\tau$ for increasing values of lower bound time $T_{\min}$, using the time-independent Hamiltonian (red) and time-dependent Hamiltonian (green) with $s_L(t)$ and evaluated for a $Cy(51)$. We see that for times smaller than a characteristic time $T^*$ the time-independent approach performs slightly better, while for large time the time-dependent one performs significantly better.}
  \label{fig:tau_increasing_time}
\end{figure}
