%%
%DISTRIBUTION OF T/P SHOWING THAT THE MIN IS FOR SMALLEST T
%%

\begin{figure}[ht]
\centering
  \begin{tikzpicture}
    \begin{axis}[name=plot,
    xmin=0, xmax=51,
    ymin=0, ymax=800,
    width = 130mm,
    xlabel = $\bm{T}$,
    ylabel = $\bm{T/p}$,
    legend style={at={(0.05,0.95)},
    anchor=north west}]
    \addplot[color=verdescuro,no marks, very thick, x=a, y=b] table{./Data/fig8/51_lin_gamma12.txt};\addlegendentry{$s_L(t)$ $\gamma=1.2$}
    \addplot[color=verdescuro,dashed, no marks,very thick, x=a, y=b] table{./Data/fig8/51_lin_gamma043.txt};\addlegendentry{\hspace{40pt}$\gamma=0.42$}
    \addplot[color=arancio,no marks, very thick, x=a, y=b] table{./Data/fig8/51_cerf_gamma13.txt};\addlegendentry{$s_{RC}(t)$ $\gamma=1.3$}
    \addplot[color=arancio,dashed, no marks,very thick, x=a, y=b] table{./Data/fig8/51_cerf_gamma076.txt};\addlegendentry{\hspace{40pt}$\gamma=0.76$}
    \addlegendimage{legend image with text=}
    \addlegendentry{$\hspace{10pt}$}
    \addlegendimage{legend image with text=}
    \addlegendentry{$\mbox{time independent}$}

    \addplot[color=rossoscuro,no marks, very thick, x=a, y=b] table{./Data/fig8/51_gamma085.txt};\addlegendentry{\hspace{40pt}$\gamma=0.85$}
    \addplot[color=rossoscuro,dashed, no marks,very thick, x=a, y=b] table{./Data/fig8/51_gamma055.txt};\addlegendentry{\hspace{40pt}$\gamma=0.55$}
    \addplot[color=rossoscuro,dotted, no marks,very thick, x=a, y=b] table{./Data/fig8/51_gamma03.txt};\addlegendentry{\hspace{40pt}$\gamma=0.30$}

    \end{axis}
  \end{tikzpicture}


  \caption{\textbf{Distribution of \bm{$T/p$} for sampled \bm{$\gamma$} showing that \bm{$\tau$} will always be for the smallest \bm{$T$}}. The figure shows the distribution of $T/p$ for some sampled values of the $\gamma$ parameter, using the linear $s_L$ (green) and Roland-Cerf $s_{RC}$ (orange) interpolating schedules for the time-dependent Hamiltonian and the time-independent Hamiltonian (red). It is clear that $\tau = \min(T/p)$ will always be for the smallest $T$ available, regardless of the interpolating schedule and the type of Hamiltonian, requiring therefore a constrain on the time.}
  \label{fig:time_independent_sampled_gamma}
\end{figure}
