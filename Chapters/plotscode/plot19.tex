%%
%COMPLETE GRAPH HEATMAP PLOTS - ROLAND-CERF and LINEAR
%%

\begin{figure}[ht]
  \centering
  \subfloat[][$s_{NL}(t)$]
  {
    \hspace{-0.8cm}
    \begin{tikzpicture}
      \begin{axis}[name=plot,
      zmin=0,zmax=1,
      view={0}{90},
      colormap={inferno}{rgb255=(248,251,155) rgb255=(251,179,21) rgb255=(236,103,38) rgb255=(187,54,84) rgb255=(119,29,109) rgb255=(49,9,92) rgb255=(1,1,8)},
      colorbar,
      colorbar right,
      point meta min=0,
      point meta max=1,
      width = 65mm,
      height= 65mm,
      xlabel = $\bm{T}$,
      ylabel = $\bm{\gamma}$
      ]
      \addplot3 [surf] table[x=a,y=b,z=c]{./Data/fig17/51_cerf_complete.txt};
      \pgfplotsset{contour/every contour label/.style = {
                 sloped,
                 inner sep=2pt,
                 transform shape,
                 every node/.style={mapped color,fill=none},
                },}
      \addplot3[
              contour gnuplot=
              {
              draw color=white,
              levels={0.9},
              labels=true,
              contour label style={nodes={text=white,font={\large}, yshift=+.4ex, xshift=-3.0ex}},
              handler/.style=smooth},
              contour/label distance=300pt,
              line width=1.5pt,
              contour/labels over line
             ] table [
                      x = a,
                      y = b,
                      z = c,
                     ]{./Data/fig17/51_cerf_complete.txt};

      \end{axis}
    \end{tikzpicture}
  }
  \subfloat[][$s_L(t)$]
  {
    \begin{tikzpicture}
      \begin{axis}[name=plot,
      zmin=0,zmax=1,
      view={0}{90},
      colormap={inferno}{rgb255=(248,251,155) rgb255=(251,179,21) rgb255=(236,103,38) rgb255=(187,54,84) rgb255=(119,29,109) rgb255=(49,9,92) rgb255=(1,1,8)},
      point meta min=0,
      point meta max=1,
      width = 65mm,
      height= 65mm,
      xlabel = $\bm{T}$,
      ylabel = $\bm{\gamma}$,
      ylabel near ticks, yticklabel pos=right
      ]
      \addplot3 [surf] table[x=a,y=b,z=c]{./Data/fig17/51_linear_complete.txt};
      \pgfplotsset{contour/every contour label/.style = {
                 sloped,
                 inner sep=2pt,
                 transform shape,
                 every node/.style={mapped color,fill=none},
                },}
      \addplot3[
              contour gnuplot=
              {
              draw color=white,
              levels={0.5},
              labels=true,
              contour label style={nodes={text=white,font={\large}, yshift=-2.5ex, xshift=3.0ex}},
              handler/.style=smooth},
              contour/label distance=300pt,
              line width=1.5pt,
              contour/labels over line
             ] table [
                      x = a,
                      y = b,
                      z = c,
                     ]{./Data/fig17/51_linear_complete.txt};
      \end{axis}
    \end{tikzpicture}
  }

  \caption[]{\textbf{Probability distributions for the complete graph \bm{$C(51)$} with the time-dependent Hamiltonian.} The figure shows the probability distribution for a complete graph of $N=51$ using the time-dependent Hamiltonian with the non-linear $s_{NL}(t)$ (a) and linear $s_L(t)$ (right) interpolating schedules. It illustrates the great impact of the interpolating schedule on the overall performance of the time-dependent algorithm, where the choice of $s(t)$ is critical.}
  \label{fig:heatmap-complete}
\end{figure}
