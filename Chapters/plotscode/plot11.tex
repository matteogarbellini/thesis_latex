%%
%TAU DISTRIBUTION FOR DIFFERENT SHAPES OF S(T)
%%
\begin{figure}[ht]
  \centering
  \begin{tikzpicture}
    \begin{axis}[name=plot,
    xmin=0, xmax=80,
    ymin=0, ymax=200,
    width = 90mm,
    xlabel = Dimension \bm{$N$},
    ylabel = $\bm{\tau}$,
    domain=0:80,
    legend style={at={(0.05,0.95)},
    anchor=north west}]

    \addplot[color=bluscuro,mark=*,mark size=2.5pt, x=a,y=b] table{./Data/fig11/tau_cbrt.txt};\addlegendentry{$s_{C}(t)$}
    \addplot[color=azzurroscuro,mark=*,mark size=2.5pt, x=a,y=b] table{./Data/fig11/tau_sqrt.txt};\addlegendentry{$s_{S}(t)$}
    \addplot[color=verdescuro,mark=*,mark size=2.5pt, x=a,y=b] table{./Data/fig11/tau_linear.txt};\addlegendentry{$s_{L}(t)$}
    \addplot[color=arancio,mark=*,mark size=2.5pt, x=a,y=b] table{./Data/fig11/tau_cerf.txt};\addlegendentry{$s_{NL}(t)$}

    \end{axis}
  \end{tikzpicture}
  \caption{\textbf{\bm{$\tau$} with constrained time at \bm{$T=\pi/2\sqrt{N}$}, for the different intepolating schedules.} The plot shows $\tau$ for cycle graphs up to $N=71$, using the time-dependent Hamiltonian with different intepolating schedules: (orange) non-linear $s_{NL}(t)$, (green) linear $s_{L}(t)$, (light blue) $s_{S}(t)$ and (blue) $s_{C}(t)$. As expected $s_{NL}(t)$ is the best performing intepolating schedule, followed by the linear $s_L(t)$. The others do not show any advantage.}
  \label{fig:delta_increasing_time}
\end{figure}
