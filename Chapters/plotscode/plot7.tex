%%
%LOCALIZATION: HEATMAP PLOT FOR TIME-DEPENDENT HAMILTONIAN UP TO LARGE T
%%

\begin{figure}[ht]
  \centering
  \begin{tikzpicture}
    \begin{axis}[name=plot,
    zmin=0,zmax=1,
    view={0}{90},
    colormap={inferno}{rgb255=(248,251,155) rgb255=(251,179,21) rgb255=(236,103,38) rgb255=(187,54,84) rgb255=(119,29,109) rgb255=(49,9,92) rgb255=(1,1,8)},
    colorbar,
    colorbar style = {ylabel=Probability, ytick={0.1,0.3,0.5,0.7, 0.9}},
    width = 120mm,
    height= 95mm,
    xlabel = $\bm{T}$,
    ylabel = $\bm{\gamma}$
    ]
    \addplot3 [surf] table[x=a,y=b,z=c]{./Data/fig7/fig7_heatmap_data.txt};

    \pgfplotsset{contour/every contour label/.style = {
               sloped,
               inner sep=2pt,
               transform shape,
               every node/.style={mapped color,fill=none},
              },}
    \addplot3[
            contour gnuplot=
            {
            draw color=white,
            levels={0.80, 0.90, 0.95, 0.99},
            labels=true,
            contour label style={nodes={text=white,font={\Large}, yshift=0.5ex, xshift=0.5ex}},
            handler/.style=smooth},
            contour/label distance=300pt,
            line width=1.5pt,
            contour/labels over line
           ] table [
                    x = a,
                    y = b,
                    z = c,
                   ]{./Data/fig7/fig7_heatmap_data.txt};

    \end{axis}
  \end{tikzpicture}
  \caption{\textbf{Localization at large T for the time-dependent Hamiltonian}. The figure shows the probability distribution for the time-dependent Hamiltonian using the linear interpolating schedule $s_L(t)$ up to large values of $T$. Although, as we can see from the white contour lines, the algorithm gets to $p>0.99$ for a large value of $T$, it is able to produce $p=0.9$ in $T/2$ and $p=0.8$ in $T/3$. }
  \label{fig:localization_linear}
\end{figure}
