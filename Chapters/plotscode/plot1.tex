%%
%INTERPOLATING SCHEDULES: ORIGINAL ROLAND-CERF AND OUR S_RC
%%
%TAU DISTRIBUTION FOR INCREASING LOWER BOUND TIME

\begin{figure}[ht]
  \centering
  \begin{tikzpicture}
    \begin{axis}[name=plot
    xmin=0, xmax=60,
    ymin=0, ymax=600,
    width = 120mm,
    xlabel = Dimension $N$,
    ylabel = $\bm{\tau}$ ]
    \addplot[color=arancio,mark=*,mark size=2.5pt, x=a ,y=b] table{./Data/51_tau_lbt_static.txt};\addlegendentry{time-independent}
    \addplot[color=rossoscuro,mark=*,mark size=2.5pt, x=a,y=b] table{./Data/51_tau_lbt.txt};\addlegendentry{$s_L(t)$}

    \end{axis}
  \end{tikzpicture}
  \caption[$\tau$ distribution for increasing lower bound time.]{\textbf{\bm{$\tau$} distribution for increasing lower bound time. }The figure shows the $\tau$ distribution for increasing values of lower bound time, using the time-independent hamiltonian (orange) and time-dependent hamiltonain (red) and evaluated for a Cy(51). We see that for times smaller than a characteristic time $T^*$ the time-independent approach performs slightly better, while for large time the time-dependent one performs significantly better.}
  \label{fig:tau_increasing_time}
\end{figure}
