%%
%I DISTRIBUTION FOR INDEPENDENT - LIN - CERF
%%

\begin{figure}[ht]
  \centering
  \begin{tikzpicture}
    \begin{axis}[name=plot,
    xmin=0, xmax=80,
    ymin=0, ymax=17,
    width = 100mm,
    xlabel = Dimension $N$,
    ylabel = $\bm{I}$,
    domain=0:80,
    legend style={at={(0.05,0.95)},
    anchor=north west}]
    \addplot[color=rossoscuro,mark=*,mark size=2.5pt, x=a,y=b] table{./Data/fig13/iters_static.txt};\addlegendentry{time-independent}
    \addplot[color=verdescuro,mark=*,mark size=2.5pt, x=a,y=b] table{./Data/fig13/iters_linear.txt};\addlegendentry{$s_{L}(t)$}
    \addplot[color=arancio,mark=*,mark size=2.5pt, x=a,y=b] table{./Data/fig13/iters_cerf.txt};\addlegendentry{$s_{RC}(t)$}
    \addplot[color=black, samples at={0,0.001,0.005,0.01,...,0.2,0.3,0.4,...,2,...,80}]{1.5*x^(1/2)};\addlegendentry{$O(\frac{2}{\pi}\sqrt{N})$}


    \end{axis}
  \end{tikzpicture}
  \caption{\textbf{\bm{$I$} with constrained time at $\bm{\pi/2\sqrt{N}}$, for the time-dependent and time-independent approaches.} The plot shows $I$ for cycle graphs up to $N=71$, using the time-independent Hamiltonian (red) and time-dependent Hamiltonian with linear $s_L$ (green) and non-linear $s_{NL}$ (orange) interpolating schedules. The black line represents a time scaling of $\sqrt{N}$. Combining the value of $I$ with the constrained time of $\pi/2\sqrt(N)$ we can estimate the performance of the algorithms: if the distribution is below the black line the algorithm performs better than the classical search $O(N)$. From this plot we can clearly see that for $N$ up to $71$ the performance is better than the classical search, although we are not able to make predictions for larger $N$.}
  \label{fig:delta_increasing_time}
\end{figure}
