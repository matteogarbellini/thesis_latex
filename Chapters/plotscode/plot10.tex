%%
%ITERS DISTRIBUTION FOR INCREASING T_MIN
%%
\begin{figure}[ht]
  \centering
  \begin{tikzpicture}
    \begin{semilogyaxis}[name=plot,
    xmin=0, xmax=60,
    ymin=0, ymax=60,
    width = 100mm,
    height= 80mm,
    xlabel = Lower bound time $T_{\min}$,
    ylabel = $\bm{I}$ ]
    \addplot[color=rossoscuro,mark=*,mark size=2.5pt, x=a ,y=b] table{./Data/fig10/51_iters_lbt_static.txt};\addlegendentry{time-independent}
    \addplot[color=verdescuro,mark=*,mark size=2.5pt, x=a,y=b] table{./Data/fig10/51_iters_lbt_linear.txt};\addlegendentry{$s_L(t)$}

    \end{semilogyaxis}
  \end{tikzpicture}
  \caption{\textbf{$\bm{I}$ for increasing lower bound time $\bm{T_{\min}}$. }The plot shows $I$ for increasing values of lower bound time $T_{\min}$, using the time-independent Hamiltonian (red) and time-dependent Hamiltonian (green) with linear $s_L(t)$ and evaluated for a Cy(51). This distribution reflects the probability distribution of the two approaches: for the time-independent Hamiltonian the probability does not increase with time, resulting in a (almost) constant $I$, while the time-dependent Hamiltonian showing localization properties requires less iterations to get to unitary probability. Note that the plot is given with logscale y axis, which helps to highlight the time-independent trend. }
  \label{fig:iters_increasing_time}
\end{figure}
