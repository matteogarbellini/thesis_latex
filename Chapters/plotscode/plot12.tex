%%
%TAU DISTRIBUTION FOR INDEPENDENT - LIN - CERF
%%
\begin{figure}[ht]
  \centering
  \begin{tikzpicture}
    \begin{axis}[name=plot,
    xmin=0, xmax=80,
    ymin=0, ymax=220,
    width = 90mm,
    xlabel = Dimension $N$,
    ylabel = $\bm{\tau}$,
    domain=0:80,
    legend style={at={(0.05,0.95)},
    anchor=north west}]
    \addplot[color=rossoscuro,mark=*,mark size=2.5pt, x=a,y=b] table{./Data/fig12/tau_static.txt};\addlegendentry{time-independent}
    \addplot[color=verdescuro,mark=*,mark size=2.5pt, x=a,y=b] table{./Data/fig12/tau_linear.txt};\addlegendentry{$s_{L}(t)$}
    \addplot[color=arancio,mark=*,mark size=2.5pt, x=a,y=b] table{./Data/fig12/tau_cerf.txt};\addlegendentry{$s_{RC}(t)$}

    \end{axis}
  \end{tikzpicture}
  \caption{\textbf{\bm{$\tau$} with constrained time at $\bm{\pi/2\sqrt{N}}$, for the time-dependent and time-independent approaches.} The plot shows $\tau$ for cycle graphs up to N=71, using the time-independent Hamiltonian (red) and time-dependent Hamiltonian with linear $s_L$ (green) and non-linear $s_{NL}$ (orange) interpolating schedules. We notice that for small graphs, up to $N\approx 20$, both approaches perform similarly. For large $N$ however the time-dependent algorithm performs significanly better.}
  \label{fig:delta_increasing_time}
\end{figure}
