%%
%T - ROBUSTNESS
%%
\begin{figure}[ht]
  \centering
  \begin{tikzpicture}
    \begin{axis}[name=plot,
    xmin=0, xmax=80,
    ymin=0, ymax=0.02,
    width = 90mm,
    xlabel = Dimension $N$,
    ylabel = $T$-robustness]
    \addplot[color=rossoscuro,mark=*,mark size=2.5pt, x=a, y=b] table{./Data/fig16/t_robustness_static.txt};\addlegendentry{time-independent}
    \addplot[color=verdescuro,mark=*,mark size=2.5pt, x=a, y=b] table{./Data/fig16/t_robustness_lin.txt};\addlegendentry{$s_{L}(t)$}
    \addplot[color=arancio,mark=*, mark size=2.5pt, x=a, y=b] table{./Data/fig16/t_robustness_cerf.txt};\addlegendentry{$s_{RC}(t)$}
    \end{axis}
  \end{tikzpicture}
  \caption[Robustness for the time-independent hamiltonian]{\textbf{$\bm{T}$-Robustness for the time-independent and time-dependent approaches.} The figure shows the $\gamma$-robustness distribution for the time-independent approach (red), the time-dependent one with linear interpolating schedule $s_L(t)$ (green) and with Roland-Cerf $s_{RC}(t)$ (red). Recalling that the lower $R$ value the highest the robustness, this distribution reflects the probability distribution seen in \Cref{subsec:time_dependent_results}, where the probability distribution was smoother for $s_L(t)$ than $s_{RC}(t)$.}
\end{figure}
