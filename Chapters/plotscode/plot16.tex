%%
%T - ROBUSTNESS
%%
\begin{figure}[ht]
  \centering
  \begin{tikzpicture}
    \begin{axis}[name=plot,
    xmin=0, xmax=80,
    ymin=0, ymax=0.017,
    width = 100mm,
    xlabel = Dimension $N$,
    ylabel = $T$-robustness]
    \addplot[color=rossoscuro,mark=*,mark size=2.5pt, x=a, y=b] table{./Data/fig16/t_robustness_static.txt};\addlegendentry{time-independent}
    \addplot[color=verdescuro,mark=*,mark size=2.5pt, x=a, y=b] table{./Data/fig16/t_robustness_lin.txt};\addlegendentry{$s_{L}(t)$}
    \addplot[color=arancio,mark=*, mark size=2.5pt, x=a, y=b] table{./Data/fig16/t_robustness_cerf.txt};\addlegendentry{$s_{NL}(t)$}
    \end{axis}
  \end{tikzpicture}
  \caption{\textbf{$\bm{T}$-Robustness for the time-independent and time-dependent approaches.} The figure shows the $T$-robustness  for the time-independent approach (red), the time-dependent one with linear $s_L(t)$ (green) and non-linear $s_{NL}(t)$ (red) interpolating schedules. Surprisingly, the time-independent approach is more robust than the time-dependent one for large $N$. However it is to be noted that the difference in values is much smaller than the one obtained for the $\gamma$-robustness.}
  \label{fig:time_robustness}
\end{figure}
